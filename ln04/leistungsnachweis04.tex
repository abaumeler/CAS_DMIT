%%%%%%%%%%%%%%%%%%%%%%%%%%%%%%%%%%%%%%%%%%%%%%%%%%%%%%%%%%%%%
%% HEADER
%%%%%%%%%%%%%%%%%%%%%%%%%%%%%%%%%%%%%%%%%%%%%%%%%%%%%%%%%%%%%
\documentclass[a4paper,oneside, 12pt]{report}
% Alternative Optionen:
%	Papiergrösse: a4paper / a5paper / b5paper / letterpaper / legalpaper / executivepaper
% Duplex: oneside / twoside
% Grundlegende Fontgrössen: 10pt / 11pt / 12pt


%% Deutsche Anpassungen %%%%%%%%%%%%%%%%%%%%%%%%%%%%%%%%%%%%%
\usepackage[ngerman]{babel}
\usepackage[T1]{fontenc}
\usepackage[utf8]{inputenc}
\usepackage[numbers]{natbib}
\usepackage{lmodern} %Type1-Schriftart für nicht-englische Texte
\usepackage{color,soul} %Highlight
\usepackage{acronym} % Abkürzungen
\usepackage[toc,page]{appendix}
\usepackage{pdfpages}
\usepackage{subcaption} %Bilder nebeneinander
\usepackage{rotating} %Tabelle Hochkant
\usepackage{color, colortbl}
\usepackage{fancyvrb} %Text File einbinden
\usepackage{embedfile}[2020/04/01]

%% Packages für Grafiken & Abbildungen %%%%%%%%%%%%%%%%%%%%%%
\usepackage{graphicx} %%Zum Laden von Grafiken
%\usepackage{subfig} %%Teilabbildungen in einer Abbildung
%\usepackage{pst-all} %%PSTricks - nicht verwendbar mit PDFLaTeX

%% Packages für Formeln %%%%%%%%%%%%%%%%%%%%%%%%%%%%%%%%%%%%%
\usepackage{amsmath}
\usepackage{amsthm}
\usepackage{amsfonts}


%% PDF-A Settings %%%%%%%%%%%%%%%%%%%%%%%%%%%%%%%%%%%%%%%%%%%
\usepackage[hyphens]{url}
\usepackage[hidelinks,pdfa]{hyperref}
\usepackage{hyperxmp}[2020/03/01]
\hypersetup{breaklinks=true}
\urlstyle{same}
\title{Leistungsnachweis 01}
\author{Andrés Baumeler}
\embedfile[afrelationship={/Source},ucfilespec={\jobname.tex},mimetype={application/x-tex}]{\jobname.tex}
\hypersetup{%
    pdflang=la,
    pdfapart=3, %set to 1 for PDF/A-1
    pdfaconformance=B
}

% %Create an OutputIntent in order to correctly specify colours
\immediate\pdfobj stream attr{/N 3} file{sRGB.icc}
\pdfcatalog{%
  /OutputIntents [
    <<
      /Type /OutputIntent
      /S /GTS_PDFA1
      /DestOutputProfile \the\pdflastobj\space 0 R
      /OutputConditionIdentifier (sRGB)
      /Info (sRGB)
    >>
  ]
}

%%%%%%%%%%%%%%%%%%%%%%%%%%%%%%%%%%%%%%%%%%%%%%%%%%%%%%%%%%%%%
%% Anmerkungen
%%%%%%%%%%%%%%%%%%%%%%%%%%%%%%%%%%%%%%%%%%%%%%%%%%%%%%%%%%%%%
%
% Zu erledigen:
% 1. Passen Sie die Packages und deren Optionen an (siehe oben).
% 2. Wenn Sie wollen, erstellen Sie eine BibTeX-Datei
%    (z.B. 'literatur.bib').
% 3. Happy TeXing!
%
%%%%%%%%%%%%%%%%%%%%%%%%%%%%%%%%%%%%%%%%%%%%%%%%%%%%%%%%%%%%%


%%%%%%%%%%%%%%%%%%%%%%%%%%%%%%%%%%%%%%%%%%%%%%%%%%%%%%%%%%%%%
%% Optionen / Modifikationen
%%%%%%%%%%%%%%%%%%%%%%%%%%%%%%%%%%%%%%%%%%%%%%%%%%%%%%%%%%%%%
% Pfad für Bilder
\graphicspath{ {../img/} }

% redefine \VerbatimInput
\RecustomVerbatimCommand{\VerbatimInput}{VerbatimInput}%
{fontsize=\footnotesize,
 %
 frame=lines,  % top and bottom rule only
 framesep=2em, % separation between frame and text
 rulecolor=\color{Gray},
 %
 label=\fbox{\color{Black}Dokumentation.txt},
 labelposition=topline,
 %
 commandchars=\|\(\), % escape character and argument delimiters for
                      % commands within the verbatim
 commentchar=!        % comment character
}

%% Zeilenabstand %%%%%%%%%%%%%%%%%%%%%%%%%%%%%%%%%%%%%%%%%%%%
\usepackage{setspace}
%\singlespacing        %% 1-zeilig (Standard)
\onehalfspacing       %% 1,5-zeilig
%\doublespacing        %% 2-zeilig

%%%%%%%%%%%%%%%%%%%%%%%%%%%%%%%%%%%%%%%%%%%%%%%%%%%%%%%%%%%%%
%% DOKUMENT
%%%%%%%%%%%%%%%%%%%%%%%%%%%%%%%%%%%%%%%%%%%%%%%%%%%%%%%%%%%%%
\begin{document}
\definecolor{Gray}{gray}{0.9}
\definecolor{LGray}{gray}{0.8}

\pagestyle{empty} %%Keine Kopf-/Fusszeilen auf den ersten Seiten.


%% Deckblatt %%%%%%%%%%%%%%%%%%%%%%%%%%%%%%%%%%%%%%%%%%%%%%%%
%% Basierend auf einer TeXnicCenter-Vorlage von Tino Weinkauf.
%%%%%%%%%%%%%%%%%%%%%%%%%%%%%%%%%%%%%%%%%%%%%%%%%%%%%%%%%%%%%%

%%%%%%%%%%%%%%%%%%%%%%%%%%%%%%%%%%%%%%%%%%%%%%%%%%%%%%%%%%%%%
%% Deckblatt
%%%%%%%%%%%%%%%%%%%%%%%%%%%%%%%%%%%%%%%%%%%%%%%%%%%%%%%%%%%%%
%%
%% ACHTUNG: Sie ben�tigen ein Hauptdokument, um diese Datei
%%          benutzen zu k�nnen. Verwenden Sie im Hauptdokument
%%          den Befehl "\input{dateiname}", um diese
%%          Datei einzubinden.
%%

\begin{titlepage}

\begin{center}
%\vspace*{1cm}
\Large
\textsc{Vergleich physischer und digitaler L�sungen zum Records
Management hinsichtlich Energie- und Raumbedarf}\\

\vspace{4cm}

%\LARGE
\textsc{Bachelorarbeit\\[0.5\baselineskip]
 von\\[0.5\baselineskip]
Andr�s Baumeler\\
{\normalsize \textsc{Matrikelnummer 11-923-331}}}\\

\vspace{3cm}
\textsc{Abgabe} \\
\textsc{19. Januar 2017}\\ %%Datum der Abgabe - am besten selbst reinschreiben.

\vspace{1cm}
\textsc{Betreuer\\
Prof. Dr. L. Hilty}\\

\vspace{2cm}
\textsc{ \includegraphics[width=4cm]{uzh_logo_d_pos.pdf} \\
Departement f�r Informatik\\
\normalsize{Informatics and Sustainability Research Group}}\\

\end{center}

\end{titlepage}



%% Inhaltsverzeichnis %%%%%%%%%%%%%%%%%%%%%%%%%%%%%%%%%%%%%%%
\cleardoublepage
\tableofcontents %Inhaltsverzeichnis
\cleardoublepage %Das erste Kapitel soll auf einer ungeraden Seite beginnen.

\pagestyle{plain} %%Ab hier die Kopf-/Fusszeilen: headings / fancy / ...


%% Kapitel / Hauptteil des Dokumentes %%%%%%%%%%%%%%%%%%%%%%%

%%%%%%%%%%%%%%%%%%%%%%%%%%%%%%%%%%%%%%%%%%
%% ==> Einleitung
\chapter{Einleitung}\label{sec:motivation}
Die rasante digitale Transformation hat zu einem exponentiellen Anstieg der digitalen Dokumente geführt, die in verschiedenen Bereichen wie Verwaltung, Wissenschaft, Bildung und Wirtschaft genutzt werden. Angesichts der Notwendigkeit einer langfristigen Aufbewahrung und Archivierung dieser Dokumente ist das \ac{PDF} zu einem der bevorzugten Formate geworden. \ac{PDF} bietet eine plattformunabhängige und konsistente Darstellung von Dokumenten auf verschiedenen Systemen und Geräten.

In diesem Bericht wird auf die verschiedenen Herausforderungen eingegangen welche sich im Bereich der digitalen Langzeitarchivierung im Zusammenhang mit dem \ac{PDF} Format ergeben. In Kapitel \ref{sec:grundlagen} werden dazu zuerst das \ac{PDF}-Format und der \ac{PDF}-Standard vorgestellt, sowie deren Entstehungsgeschichte beleuchtet. Weiter wird in diesem Kapitel ein Überblick vermittelt wie \ac{PDF} in der digitalen Langzeitarchivierung verwendet werden kann und welche Vorteile das Format bietet. 

Im Kapitel \ref{sec:herausforderungen} werden eine Auswahl von Herausforderungen in der digitalen Langzeitarchivierung vorgestellt und es wird aufgezeigt wie der PDF-Standard und das PDF Format bei der Bewältigung dieser Herausforderungen unterstützen können. Konkret geht es um die Herausforderungen "Sicherstellen der Authentizität", "Erschliessung und Zugänglichkeit" sowie die Kurzlebigkeit von Technologien.

Im Kapitel \ref{sec:schluss} wird versucht einzuschätzen, ob der PDF-Standard und das PDF Format für die digitale Langzeitarchivierung geeignet sind. Weiter wird versucht abzuschätzen wie sich diese Herausforderungen in näherer Zukunft verändern werden und welche neuen Herausforderungen auftauchen könnten.




%%%%%%%%%%%%%%%%%%%%%%%%%%%%%%%%%%%%%%%%%%
%% Grundlagen
\chapter{Grundlagen}\label{sec:grundlagen}
\section{PDF-Format und PDF-Standard}
Das PDF Format wurde von der Firma Adobe Systems entwickelt und im Jahr 1993 vorgestellt. In den folgenden Jahren wurde das Format immer bekannter und zeigte Potenzial für die digitale Langzeitarchivierung. Im Jahr 2002 wurde eine Arbeitsgruppe innerhalb der \ac{ISO} gegründet, um ein standard Format für die digitale Langzeitarchivierung zu entwickeln. Vertreter einer Vielzahl von US-amerikanischen Verbänden und Bundesbehörden, darunter AIIM (Association for Information and Image Management), NPES (Association for Suppliers of Printing, Publishing and Converting Technologies) und NARA (National Archives and Records Administration), trafen sich mit Experten aus dem Bibliothekswesen (Harvard University Libraries, Library of Congress), dem Justizsystem (Administrative Office of the United States Courts) und der Industrie (einschliesslich Adobe Systems und Kodak). Am 1. Oktober 2005 wurde der PDF/A-Standard unter der Bezeichnung ISO 19005-1:2005 (PDF/A-1) veröffentlicht. Es war das weltweit erste standardisierte Dateiformat für die digitale Langzeitarchivierung. Seitdem sind drei weitere Teile des Standards erschienen: PDF/A-2 (Open-Type Schriftarten und digitale Signaturen) und PDF/A-3 (einbetten von originalen Dateien im PDF). 2020 erschien mit PDF/A-4 (Einbetten von 3D Objekten) der jüngste Teil. \cite{pdfhist}

Der PDF/A Standard ist als mehrteilige Serie angelegt. Das bedeutet, nachfolgende Versionen verdrängen vorhergehende Versionen nicht. Der Standard PDF/A-1 ist weiterhin gültig, auch wenn mittlerweile PDF/A-2 und PDF/A-3 erschienen sind. Für die Teile PDF/A-1 bis PDF/A-3 gibt es jeweils drei Konformitätsstufen. Diese Stufen werden mit A, B, oder U beschrieben. Ein PDF ist konform auf Stufe A, wenn alle Anforderungen des PDF/A Standards erfüllt sind. Das beinhaltet unter anderem auch, dass der Text innerhalb des Dokuments in der natürlichen Lesereihenfolge angeordnet sein muss. Stufe B sagt aus, dass ein Dokument eindeutig reproduziert werden kann. Ein Dokument auf Stufe B muss aber im Gegensatz zu Stufe A nicht 100 \% Textextraktion und Durchsuchbarkeit bieten. Dokumente auf Stufe U garantieren, dass sämtlicher enthaltener Text zu standard Unicode Character Codes gemappt werden kann. \cite{pdfhist}, \cite{pdftools}, \cite{adobepdf}


\section{PDF für die Langzeitarchivierung}
Für die digitale Langzeitarchivierung ist es wichtig, dass der Inhalt von Dokumenten jederzeit und auf jeder Plattform gleich aussieht. Neben der Reproduzierbarkeit müssen Dokumente auch in Zukunft, unter Umständen auf einer heute noch nicht existierenden Plattform geöffnet werden können. Da sich nur schwer abschätzen lässt, wie sich die heute etablierten Betriebssysteme entwickeln, kann nicht vorausgesetzt werden, dass die heute verwendeten Programme auch in Zukunft noch zur Verfügung stehen oder auf den in Zukunft etablierten Betriebssystemen lauffähig sind.

Der PDF/A Standard und das darin beschriebene Dateiformat bieten für diese Herausforderungen eine Lösung. Der PDF/A Standard baut auf dem PDF Standard auf und stellt zusätzliche Anforderungen im Hinblick auf die Langzeitarchivierung. Um die Reproduzierbarkeit zu gewährleisten, muss im PDF/A Standard etwa sichergestellt sein, dass Ressourcen welche für die Darstellung des Inhalts, wie etwa Schriftarten, in die Datei eingebettet sind. Der PDF/A Standard setzt zudem voraus, dass Inhalte nicht verschlüsselt sein dürfen und dass gewisse JavaScript Funktionen nicht verwendet werden dürfen. Weiter wird vom PDF/A Standard auch verlangt, dass Metadaten im \ac{XMP} Format eingebettet werden. So lassen sich Metadaten zum Dokument in einem standardisierten Format maschinell auslesen. Damit das PDF Format auch in Zukunft noch interpretiert werden kann, ist es wichtig, dass die Spezifikation für das PDF Format offen zugänglich ist. Die \ac{ISO} stellt sicher, dass die Standards öffentlich Verfügbar sind. Durch diese Zugänglichkeit können Programme zum Erstellen und Betrachten von PDF Dokumenten auf für zukünftige Plattformen und Betriebssysteme entwickelt oder angepasst werden. So ist sichergestellt, dass PDF Dokumente auch noch betrachtet werden können, wenn ein Hersteller die Entwicklung eines Programms einstellt.\cite{pdfanutshell}, \cite{Arms2005DigitalFF}, \cite{ISO32000}, \cite{ISO19005}

 
%%%%%%%%%%%%%%%%%%%%%%%%%%%%%%%%%%%%%%%%%%
%% Herausforerungen
\chapter{Herausforderungen}\label{sec:herausforderungen}
In einer Studie der Schweizer Nationalbibliothek wurden Schweizer Gedächtnisinstitutionen zu Herausforderungen der digitalen Langzeitarchivierung befragt. In der Studie wurden eine Vielzahl Herausforderungen ermittelt. Nachfolgend wird auf drei Herausforerungen eingegangne wobei das PDF-Format konkrete Unterstützung bieten kann. Die teilnehmenden Institutionen nannten Herausforderungen bei der Sicherstellung der Authentizität, der Erschliessung und Zugänglichkeit sowie bei der Kurzlebigkeit von Technologien. Nachfolgend wird auf diese drei Bereiche eingegangen und erläutert wie der PDF-Standar und das PDF-Format bei diesen Herausforderungen helfen kann. \cite{lzaschweiz}

\section{Sicherstellung der Authentizität}
Der PDF-Standard bietet mittels digitalen Signaturen eine Möglichkeit die Authentizität eines Dokuments zu prüfen. Dabei wird nach dem Erstellen des Dokuments eine digitale Signatur am Dokument angebracht. Der PDF-Standard macht bei gewisse Vorgaben welche Signaturen erlaubt sind und wie diese anzubringen sind. Durch diese Signatur kann später festgestellt werden welche Institution das Dokument erstellt hat, und ob das Dokument nach dem Zeitpunkt der Signatur Erstellung verändert wurden. Eine digitale Signatur erlaubt es aber noch nicht die Authentizität eines Dokuments sicherzustellen. Dazu werden Prozesse innerhalb der erstellenden Organisation benötigt welche sicherstellen, dass die digitalen Signaturen korrekt erstellt und interpretiert werden. Zudem braucht es innheralb der Institution ein System um kontrollieren wer Signaturen anbringen darf und welchen Signaturen vertraut wird. Der PDF-Standard wirkt hier unterstützend indem gewisse Vorgaben für das Erstellen und Anbringen von digitalen Signaturen gemacht werden und bietet Möglichkeiten welche von anderen Prozessen und Software innerhalb der Institution verwendet werden können.

Das Problem der Sicherstellung der Authentizität wird noch verschärft, wenn Dokumente archiviert werden sollen, welche nicht innerhalb der eigenen Organisation erstellt wurden. Für die Archivierung solcher Fremddokumente stellt die digitale Signatur ebenfalls nur ein Bauteil dar. Es werden weiter Prozesse und Schnittstellen benötigt, um die Authentizität von Fremddokumenten sicherzustellen. \cite{pdftools}, \cite{pdfanutshell}



\section{Erschliessung und Zugänglichkeit}
 Eine zentrale Herausforderung besteht in der Gewährleistung der langfristigen Lesbarkeit und Interpretierbarkeit von \ac{PDF}-Dokumenten. Durch die kontinuierliche Weiterentwicklung des \ac{PDF}-Formats und die Einführung neuer Versionen besteht das Risiko, dass ältere Versionen möglicherweise nicht mehr von zukünftiger Software und Hardware unterstützt werden. Diese Herausforderung kann nur teilweise durch das PDF-Format gelöst werden. Durch die \ac{ISO} wird sichergestellt, dass der Standard langfristig weiterentwickelt wird und offen verfügbar ist. Dennoch liegt es hier in der Verantwortung der Institution geeignete Prozesse einzuführen um die verwendeten Formate und Software regelmässig auf ihre Eignung zu prüfen und gegebenenfalls zu aktualisieren. Nur so kann sichergestellt werden, das archivierte Inhalte auf den gewünschten Plattformen und Schnittstellen zugänglich bleiben. 

 Bei der Erschliessung bietet das PDF Format Unterstützung, indem es die Erfassung von Metadaten direkt im Dokument im \ac{XMP} Format erlaubt. Dadurch kann eine langfristige Erschliessung der Inhalte durch das PDF Format unterstützt werden. Die Erfassung der Metadaten in einem File reicht aber nicht aus, um eine wirklich Langfristige Erschliessung sicherzustellen - dazu werden geeignete Prozesse und Software welche diese Prozesse unterstützen in den Institutionen vorausgesetzt, welche das PDF Format für die digitale Langzeitarchivierung verwenden.

 Der PDF/A-3 Standard erlaubt es jedes Dateiformat in ein PDF Dokument einzubetten. Diese Funktion kann etwa verwendet werden, um die Originaldatei, aus welcher ein PDF generiert wurde in ein PDF einzubetten. Der PDF-Standard macht aber keine Angaben darüber, ob die eingebetteten Dateien für die digitale Langzeitarchivierung geeignet sein müssen. Durch ein fehlendes Verständnis des PDF/A-3 Standards kann es vorkommen, dass eine Institution fälschlicherweise davon ausgeht, dass Dokumente für die langfristige Archivierung geeignet sind, nur weil diese PDF/A-3 kompatibel sind. Dadurch kann es einerseits bei der Erschliessung zu Problemen kommen, da die eingebetteten Dokumente nicht sauber erschlossen werden, sondern nur das umfassende PDF Dokument. Andererseits kann es durch das Einbetten von Dokumenten auch zu Problemen bei der Zugänglichkeit kommen, wenn in Zukunft eingebettete Dateien nicht mehr interpretiert werden können, da die dafür notwendige Software nicht mehr existiert. \cite{pdftools}, \cite{pdfanutshell}

\section{Kurzlebigkeit von Technologien}
Grundsätzlich ist das PDF Format eine Technologie wie andere Dateiformate auch in läuft deshalb auch Gefahr in Vergessenheit zu geraten. Was das PDF Format aber von anderen Formaten unterscheidet ist, dass hier die \ac{ISO} dahintersteht und sicherstellt, dass die Spezifikation des Formats auch in Zukunft noch öffentlich verfügbar ist und weiterentwickelt wird. 

Institutionen können aber trotz Einsatz des PDF-Formats von kurzlebigen Technologien betroffen sein, etwa wenn für die Erstellung von PDF Dateien spezifische Software vorausgesetzt wird, oder wenn die Verwaltung von PDF Dateien in einem Archiv- oder Dokumentenmanagementsystem erfolgt. Dabei kann es vorkommen dass diese Software obsolet wird und die Institution auf einem unorganisierten aber lesbaren Haufen PDF Dateien sitzen bleibt.

Durch den Einsatz von PDF Formaten, insbesondere PDF/A kann bei dieser Herausforderung insofern Abhilfe geschafft werden, als das eine allfällige Migration eines Archiv- oder Dokumententenmanagementsystems vereinfacht wird, da PDF Dokumente von einer Vielzahl verschiedener Programme gelesen und bearbeitet werden können. Werden Dokumente in einem PDF/A konformen Format aufbewahrt, sollte es auch keine Probleme mit Abhängigkeiten zu externen Ressourcen geben welche eine Migration erschweren könnten.

Der PDF-Standard und das PDF Format sind auch für diese Herausforderung keine direkte Lösung sondern bieten ledgiglich Unterstützung. Es werden etablierte Prozesse wie etwa ein Preservation Planning innerhalb der Institutionen vorausgesetzt, um eine digitale Langzeitarchivierung zu gewährleisten. \cite{pdfaorgsponsored}


%%%%%%%%%%%%%%%%%%%%%%%%%%%%%%%%%%%%%%%%%%
%% ==> Schluss
\chapter{Schluss}\label{sec:schluss}
\section{Einschätzung}
Der PDF/A Standard und das darin beschriebene PDF Dateiformat sind eine geeignete Lösung für die digitale Langzeitarchivierung von Inhalten welche sich auf geeignete Art und Weise als Dokument reproduzieren lassen. Für Inhalte wie Bilder oder Film ist das Format nicht geeignet. Als grossen Vorteil des Formats sehe ich, dass mit der \ac{ISO} eine Organisation hinter dem Format steht, welche in der Lage ist die notwendige Langfristigkeit und Offenheit des Standards sicherzustellen welche für die Verwendung in der Archivierung vorausgesetzt wird. Der Einsatz von PDF/A ist aber kein Allerheilmittel, sondern lediglich ein Bauteil in einem komplexeren Gebilde. Neben der Wahl der Dateiformate gehören auch eine sorgfältige Planung und eine laufende Überprüfung der getroffenen Annahmen zu einer seriösen digitalen Langzeit Archivierung. 

Meiner Einschätzung nach ist das PDF/A Format eine solide Entscheidung für die digitale Langzeitarchivierung wenn dies durch geeignete Prozesse und Software ergänzt wird und die getroffenen Annahmen periodisch neu verifiziert werden. 

Weiter ist es unerlässlich, die verschiedenen PDF Standards genau zu verstehen, vorallem im Hinblick darauf was die verschiedenen Konformitätslevel aussagen. Denn alleine durch eine Konvertierung zu PDF/A-1b ist eine langfristige Lesbarkeit und Reproduzierbarkeit noch nicht gewährleistet.

\section{Ausblick}
Die digitale Langzeitarchivierung muss sich stetig anpassen, da immer neue Inhalte mit neuen Eigenschaften und Strukturen aufbewahrt werden müssen. Das bedeutet auch, dass der PDF-Standard weiterentwickelt werden muss, um zukünftigen Entwicklungen gerecht zu werden. In der nahen Zukunft werden immer mehr und verschiedenartige Daten produziert werden, welche zu PDF konvertiert und aufbewahrt werden sollen. Hierbei wird der PDF-Standard weiterhin einen wertvollen Beitrag leisten können. Durch die Veröffentlichung von weiteren Standards wie PDF/A-4, PDF/X-4 und PDF/VT wurden in der näheren Vergangenheit bereits neue PDF Standards veröffentlicht, um sich ändernden Anforderungen gerecht zu werden. Diese Entwicklung wird sich meiner Ansicht nach auch so fortsetzen. Durch Fortschritte in der Forschung wird es wohl dazu kommen, dass der Standard um weitere Kompressionsalgorithmen und Bildformate ergänzt wird, um mit den steigenden Dateigrössen Schritt zu halten.

%%%%%%%%%%%%%%%%%%%%%%%%%%%%%%%%%%%%%%%%%%%%%%%%%%%%%%%%%%%%%
%% LITERATUR UND ANDERE VERZEICHNISSE
%%%%%%%%%%%%%%%%%%%%%%%%%%%%%%%%%%%%%%%%%%%%%%%%%%%%%%%%%%%%%
%% Ein kleiner Abstand zu den Kapiteln im Inhaltsverzeichnis (toc)
\addtocontents{toc}{\protect\vspace*{\baselineskip}}

%% Abkürzungen
\cleardoublepage
\phantomsection
\addcontentsline{toc}{chapter}{Abkürzungen}
\chapter*{Abkürzungen}
\begin{acronym}[Abkürzungen]
\acro{PDF}{Portable Document Format}
\acro{PDF/A}{Portable Document Format for Archiving}
\acro{ISO}{International Organization for Standardization}
\acro{XMP}{Extensible Metadata Platform}
\end{acronym}

%% Literaturverzeichnis %%%%%%%%%%%%%%%%%%%%%%%%%%%%%%%%%%%%%%
%% ==> Eine Datei 'literatur.bib' wird hierfür benötigt.
\cleardoublepage
\phantomsection
\addcontentsline{toc}{chapter}{Literaturverzeichnis}
%\nocite{*} %Auch nicht-zitierte BibTeX-Einträge werden angezeigt.
\Urlmuskip=0mu plus 1mu\relax
\bibliographystyle{plainnat} %Art der Ausgabe: plain / apalike / amsalpha / ...
\bibliography{literatur} %Eine Datei 'literatur.bib' wird hierfür benötigt.
\end{document}