%%%%%%%%%%%%%%%%%%%%%%%%%%%%%%%%%%%%%%%%%%%%%%%%%%%%%%%%%%%%%
%% HEADER
%%%%%%%%%%%%%%%%%%%%%%%%%%%%%%%%%%%%%%%%%%%%%%%%%%%%%%%%%%%%%
\documentclass[a4paper,oneside, 12pt]{report}
% Alternative Optionen:
%	Papiergrösse: a4paper / a5paper / b5paper / letterpaper / legalpaper / executivepaper
% Duplex: oneside / twoside
% Grundlegende Fontgrössen: 10pt / 11pt / 12pt


%% Deutsche Anpassungen %%%%%%%%%%%%%%%%%%%%%%%%%%%%%%%%%%%%%
\usepackage[ngerman]{babel}
\usepackage[T1]{fontenc}
\usepackage[utf8]{inputenc}
\usepackage[numbers]{natbib}
\usepackage{lmodern} %Type1-Schriftart für nicht-englische Texte
\usepackage{color,soul} %Highlight
\usepackage{acronym} % Abkürzungen
\usepackage[toc,page]{appendix}
\usepackage{pdfpages}
\usepackage{subcaption} %Bilder nebeneinander
\usepackage{rotating} %Tabelle Hochkant
\usepackage{color, colortbl}
\usepackage{fancyvrb} %Text File einbinden
\usepackage{embedfile}[2020/04/01]

%% Packages für Grafiken & Abbildungen %%%%%%%%%%%%%%%%%%%%%%
\usepackage{graphicx} %%Zum Laden von Grafiken
%\usepackage{subfig} %%Teilabbildungen in einer Abbildung
%\usepackage{pst-all} %%PSTricks - nicht verwendbar mit PDFLaTeX

%% Packages für Formeln %%%%%%%%%%%%%%%%%%%%%%%%%%%%%%%%%%%%%
\usepackage{amsmath}
\usepackage{amsthm}
\usepackage{amsfonts}


%% PDF-A Settings %%%%%%%%%%%%%%%%%%%%%%%%%%%%%%%%%%%%%%%%%%%
\usepackage[hyphens]{url}
\usepackage[hidelinks,pdfa]{hyperref}
\usepackage{hyperxmp}[2020/03/01]
\hypersetup{breaklinks=true}
\urlstyle{same}
\title{Leistungsnachweis 01}
\author{Andrés Baumeler}
\embedfile[afrelationship={/Source},ucfilespec={\jobname.tex},mimetype={application/x-tex}]{\jobname.tex}
\hypersetup{%
    pdflang=la,
    pdfapart=3, %set to 1 for PDF/A-1
    pdfaconformance=B
}

% %Create an OutputIntent in order to correctly specify colours
\immediate\pdfobj stream attr{/N 3} file{sRGB.icc}
\pdfcatalog{%
  /OutputIntents [
    <<
      /Type /OutputIntent
      /S /GTS_PDFA1
      /DestOutputProfile \the\pdflastobj\space 0 R
      /OutputConditionIdentifier (sRGB)
      /Info (sRGB)
    >>
  ]
}

%%%%%%%%%%%%%%%%%%%%%%%%%%%%%%%%%%%%%%%%%%%%%%%%%%%%%%%%%%%%%
%% Anmerkungen
%%%%%%%%%%%%%%%%%%%%%%%%%%%%%%%%%%%%%%%%%%%%%%%%%%%%%%%%%%%%%
%
% Zu erledigen:
% 1. Passen Sie die Packages und deren Optionen an (siehe oben).
% 2. Wenn Sie wollen, erstellen Sie eine BibTeX-Datei
%    (z.B. 'literatur.bib').
% 3. Happy TeXing!
%
%%%%%%%%%%%%%%%%%%%%%%%%%%%%%%%%%%%%%%%%%%%%%%%%%%%%%%%%%%%%%


%%%%%%%%%%%%%%%%%%%%%%%%%%%%%%%%%%%%%%%%%%%%%%%%%%%%%%%%%%%%%
%% Optionen / Modifikationen
%%%%%%%%%%%%%%%%%%%%%%%%%%%%%%%%%%%%%%%%%%%%%%%%%%%%%%%%%%%%%
% Pfad für Bilder
\graphicspath{ {../img/} }

% redefine \VerbatimInput
\RecustomVerbatimCommand{\VerbatimInput}{VerbatimInput}%
{fontsize=\footnotesize,
 %
 frame=lines,  % top and bottom rule only
 framesep=2em, % separation between frame and text
 rulecolor=\color{Gray},
 %
 label=\fbox{\color{Black}Dokumentation.txt},
 labelposition=topline,
 %
 commandchars=\|\(\), % escape character and argument delimiters for
                      % commands within the verbatim
 commentchar=!        % comment character
}

%% Zeilenabstand %%%%%%%%%%%%%%%%%%%%%%%%%%%%%%%%%%%%%%%%%%%%
\usepackage{setspace}
%\singlespacing        %% 1-zeilig (Standard)
\onehalfspacing       %% 1,5-zeilig
%\doublespacing        %% 2-zeilig

%%%%%%%%%%%%%%%%%%%%%%%%%%%%%%%%%%%%%%%%%%%%%%%%%%%%%%%%%%%%%
%% DOKUMENT
%%%%%%%%%%%%%%%%%%%%%%%%%%%%%%%%%%%%%%%%%%%%%%%%%%%%%%%%%%%%%
\begin{document}
\definecolor{Gray}{gray}{0.9}
\definecolor{LGray}{gray}{0.8}

\pagestyle{empty} %%Keine Kopf-/Fusszeilen auf den ersten Seiten.


%% Deckblatt %%%%%%%%%%%%%%%%%%%%%%%%%%%%%%%%%%%%%%%%%%%%%%%%
%% Basierend auf einer TeXnicCenter-Vorlage von Tino Weinkauf.
%%%%%%%%%%%%%%%%%%%%%%%%%%%%%%%%%%%%%%%%%%%%%%%%%%%%%%%%%%%%%%

%%%%%%%%%%%%%%%%%%%%%%%%%%%%%%%%%%%%%%%%%%%%%%%%%%%%%%%%%%%%%
%% Deckblatt
%%%%%%%%%%%%%%%%%%%%%%%%%%%%%%%%%%%%%%%%%%%%%%%%%%%%%%%%%%%%%
%%
%% ACHTUNG: Sie ben�tigen ein Hauptdokument, um diese Datei
%%          benutzen zu k�nnen. Verwenden Sie im Hauptdokument
%%          den Befehl "\input{dateiname}", um diese
%%          Datei einzubinden.
%%

\begin{titlepage}

\begin{center}
%\vspace*{1cm}
\Large
\textsc{Vergleich physischer und digitaler L�sungen zum Records
Management hinsichtlich Energie- und Raumbedarf}\\

\vspace{4cm}

%\LARGE
\textsc{Bachelorarbeit\\[0.5\baselineskip]
 von\\[0.5\baselineskip]
Andr�s Baumeler\\
{\normalsize \textsc{Matrikelnummer 11-923-331}}}\\

\vspace{3cm}
\textsc{Abgabe} \\
\textsc{19. Januar 2017}\\ %%Datum der Abgabe - am besten selbst reinschreiben.

\vspace{1cm}
\textsc{Betreuer\\
Prof. Dr. L. Hilty}\\

\vspace{2cm}
\textsc{ \includegraphics[width=4cm]{uzh_logo_d_pos.pdf} \\
Departement f�r Informatik\\
\normalsize{Informatics and Sustainability Research Group}}\\

\end{center}

\end{titlepage}



%% Inhaltsverzeichnis %%%%%%%%%%%%%%%%%%%%%%%%%%%%%%%%%%%%%%%
\cleardoublepage
\tableofcontents %Inhaltsverzeichnis
\cleardoublepage %Das erste Kapitel soll auf einer ungeraden Seite beginnen.

\pagestyle{plain} %%Ab hier die Kopf-/Fusszeilen: headings / fancy / ...


%% Kapitel / Hauptteil des Dokumentes %%%%%%%%%%%%%%%%%%%%%%%

%%%%%%%%%%%%%%%%%%%%%%%%%%%%%%%%%%%%%%%%%%
%% ==> Einleitung
\chapter{Einleitung}\label{sec:motivation}
Die rasante digitale Transformation hat zu einem exponentiellen Anstieg der digitalen Dokumente geführt, die in verschiedenen Bereichen wie Verwaltung, Wissenschaft, Bildung und Wirtschaft genutzt werden. Angesichts der Notwendigkeit einer langfristigen Aufbewahrung und Archivierung dieser Dokumente ist das \ac{PDF} zu einem der bevorzugten Formate geworden. \ac{PDF} bietet eine plattformunabhängige und konsistente Darstellung von Dokumenten auf verschiedenen Systemen und Geräten.

In Kapitel \ref{sec:grundlagen} wird das \ac{PDF}-Format und der \ac{PDF}-Standard vorgestellt, sowie einen Überblick über die Anwendung von \ac{PDF} in der digitalen Langzeitarchivierung gegeben. 

Im Kapitel \ref{sec:herausforderungen} wird auf eine Auswahl von Herausforderungen eingegangen welche sich durch die Verwendung des \ac{PDF}-Formates in der Langzeitarchivierung ergeben.

Im Kapitel \ref{sec:schluss} findet eine Einschätzung der vorgestellten Herausforderungen statt. Weiter wird versucht abzuschätzen wie sich diese Herausforderungen in näherer Zukunft verändern werden.




%%%%%%%%%%%%%%%%%%%%%%%%%%%%%%%%%%%%%%%%%%
%% Grundlagen
\chapter{Grundlagen}\label{sec:grundlagen}
\section{PDF Format und PDF Standard}
Das PDF Format wurde von Adobe Systems entwickelt und im Jahr 1993 vorgestellt. In den folgenden Jahren wurde das Format immer bekannter und zeigte Potenzial für die digitale Langzeitarchivierung. Im Jahr 2002 wurde eine Arbeitsgruppe innerhalb der \ac{ISO} gegründet, um ein standard Format für die digitale Langzeitarchivierung zu entwickeln. Vertreter einer Vielzahl von US-amerikanischen Verbänden und Bundesbehörden, darunter AIIM (Association for Information and Image Management), NPES (Association for Suppliers of Printing, Publishing and Converting Technologies) und NARA (National Archives and Records Administration), trafen sich mit Experten aus dem Bibliothekswesen (Harvard University Libraries, Library of Congress), dem Justizsystem (Administrative Office of the United States Courts) und der Industrie (einschließlich Adobe Systems und Kodak). Am 1. Oktober 2005 wurde der PDF/A-Standard unter der Bezeichnung ISO 19005-1:2005 (PDF/A-1) veröffentlicht. Es war das weltweit erste standardisierte Dateiformat für die digitale Langzeitarchivierung. Seitdem sind drei weitere Teile des Standards erschienen: PDF/A-2 (Open-Type Schriftarten und digitale Signaturen) und PDF/A-3 (einbetten von original Dateien im PDF). 2020 erschien mit PDF/A-4 (Einbetten von 3D Objekten) der jüngste Teil. \cite{pdfhist}

Der PDF/A Standard ist als mehrteilige Serie angelegt. Das bedeutet, nachfolgende Versionen verdrängen vorhergehende Versionen nicht. Der Standard PDF/A-1 ist weiterhin gültig auch wenn mittlerweile PDF/A-2 und PDF/A-3 erschienen sind. Für die Teile PDF/A-1 bis PDF/A-3 gibt es jeweils drei Konformitätsstufen. Diese Stufen werden mit A, B, oder U beschrieben. Ein PDF ist konform auf Stufe A wenn alle Anforderungen des PDF/A Standards erfüllt sind. Das beinhaltet unteranderem auch, dass der Text innerhalb des Dokuments in der natürlichen Lesereihenfolge angeordnet sein muss. Stufe B sagt aus, dass ein Dokument eindeutig reproduziert werden kann. Ein Dokument auf Stufe B muss aber im Gegensatz zu Stufe A nicht 100\% Textextraktion und Durchsuchbarkeit bieten. Dokumente auf Stufe U garantieren, dass sämtlicher enthaltener Text zu standart Unicode Character Codes gemappt werden kann. \cite{pdfhist}, \cite{pdftools}


\section{PDF für die Langzeitarchivierung}
Für die digitale Langzeitarchivierung ist es wichtig, dass der Inhalt von Dokumenten jederzeit und auf jeder Plattform gleich aussieht. Neben der Reproduzierbarkeit müssen Dokumente auch in Zukunft noch geöffnet werden können. Da sich nur schwer abschätzen lässt, wie sich die heute etablierten Betriebssysteme entwickeln, kann nicht vorausgessetzt werden, dass die heute verwendeten Programme auch in Zukunft noch zur Verfügung stehen oder auf den in Zukunft etablierten Betriebsystemen lauffähig sind.

Der PDF/A Standard und das darin beschriebene Dateiformat bieten für diese Herausforderungen eine Lösung. Um die Reproduzierbarkeit sicherzustellen, setzt der PDF/A Standard voraus, dass Ressourcen welche für die Darstellung des Inhalts, wie etwa Schriftarten, in die Datei eingebetet sein müssen. Der PDF/A Standard setzt zudem voraus, dass Inhalte nicht verschlüsselt sein dürfen und dass gewisse Java Script Funktionen nicht verwendet werden dürfen. Weiter wird vom PDF/A Standard auch verlangt, dass Metadaten im XMP Format eingebettet werden. Damit das PDF Format auch in Zukunft noch interpretiert werden kann, ist es wichtig, dass die Spezifikation für das PDF Format offen zugänglich ist. Die \ac{ISO} stellt sicher, dass die Standards öffentlich Verfügbar sind. Durch diese Zugänglichkeit können Programme zum Erstellen und Betrachten von PDF Dokumenten auf für zukünftige Platformen und Betriebssysteme entwicklet oder angepasst werden. So ist sichergestellt, dass PDF Dokumente auch noch betrachtet werden können, wenn ein Hersteller die Entwicklung eines Programs einstellt.\cite{pdfanutshell}

 
%%%%%%%%%%%%%%%%%%%%%%%%%%%%%%%%%%%%%%%%%%
%% Herausforerungen
\chapter{Herausforderungen}\label{sec:herausforderungen}
In einer Studie der Schweizer Nationalbibliothek wurden Schweizer Gedächtnisinstitutionen zu Herausforderungen der digitalen Langzeitarchivierung befragt. Dabei wurden verschiedene Bereiche genannt bei welchen das PDF Format Unterstützung bieten kann. Die Teilnehmenden Institutionen nannten unteranderem Herausforderungen bei der Sicherstellung der Authentitiztät, der Erschliessung und der Zugänglichkeit.\cite{lzaschweiz}

\section{Sicherstellung der Authentizität}
Der PDF Standard bietet mittels digitalen Signaturen eine Möglichkeit die Authentizität eines Dokuments zu prüfen. Eine digitale Signatur erlaubt es aber noch nicht die Authentizität eines Dokuments sicherzustellen. Dazu werden Prozesse innerhalb der erstellenden Organisation benötigt welche sicherstellen, dass die digitalen Signaturen korrekt erstellt und interpretiert werden. Der PDF Standard wirkt auch hier unterstützend und bietet Möglichkeiten welche von anderen Prozessen und Software innerhalb der Institution verwendet werden können.

\section{Erschliessung und Zugänglichkeit}
 Eine zentrale Herausforderung besteht in der Gewährleistung der langfristigen Lesbarkeit und Interpretierbarkeit von \ac{PDF}-Dokumenten. Durch die kontinuierliche Weiterentwicklung des \ac{PDF}-Formats und die Einführung neuer Versionen besteht das Risiko, dass ältere Versionen möglicherweise nicht mehr von zukünftiger Software und Hardware unterstützt werden. Diese Herausforderung kann nur teilweise durch das PDF Format gelöst werden. Durch die \ac{ISO} wird sichergestellt, dass der Standard langfristig weiterentwickelt wird und offen verfügbar ist. Denoch liegt es hier in der Verantwortug der Institution geeignete Prozesse einzuführen um die verwendeten Formate und Software regelmässig auf ihre Eignung zu Prüfen und gegebenfalls zu aktualisieren. Nur so kann sichergestellt werden, das archivierte Inhalte auf den gewünschten Platformen und Schnittstellen zugänglich bleiben. 

 Bei der Erschliessung bietet das PDF Format Unterstützung indem es die Erfassung von Metadaten direkt im Dokument im XMP Format erlaubt. Dadurch kann eine langfristige Erschliessung der Inhalte durch das PDF Format unterstützt werden. Die Erfassung der Metadaten in einem File reicht aber nicht aus um eine wirklich Langfristige Erschliessung sicherzustellen - dazu werden geeignete Prozesse und Software welche diese Prozesse unterstützen in den Institutionen vorausgesetzt welche das PDF Format für die digitale Langzeitarchivierung verwenden.

\section{Eingebettete Objekte und Interaktivität}
Der PDF/A-3 Standart erlaubt es jedes Fileformat in ein PDF Dokument einzubetten. Diese Funktion kann etwa verwendet werden um die original Datei aus welcher ein PDF generiert wurde in ein PDF einzubetten. Der PDF Standard macht aber keine Angaben darüber ob die eingebeteten Dateien für die digitale Langzeitarchivierung geeignet sein müssen. Durch ein fehlendes Verständnis des PDF/A-3 Standards kann es vorkommen, dass eine Institution fälschlicherweise davon ausgeht, dass Dokumente für die langfristige Archivierung geeignet sind, nur weil diese im PDF/A-3 Format vorliegen.


%%%%%%%%%%%%%%%%%%%%%%%%%%%%%%%%%%%%%%%%%%
%% ==> Schluss
\chapter{Schluss}\label{sec:schluss}
\section{Einschätzung}
% PDF ist gut geeignet für die Archivierung
% PDF ist weit verbreitet für die Archivierung, was dafür sorgt dass das Wissen über das Format erhalten bleibt.
% Sorgfältige Planung und laufende Überprüfung ist notwendig für eine seriöse Langzeit archivierung
Der PDF/A Standard und das darin beschriebene PDF Dateiformat sind eine geeignete Lösung für die digitale Langzeitarchivierung von Inhalten welche sich auf geeigntete Art und Weise als Dokument reproduzieren lassen, auch wenn es organisatorische, technische oder finanzielle Herausforderungen bei der Verwendung des PDF Formats entstehen können. Mit der \ac{ISO} steht zudem eine Organisation hinter dem Format welche in der Lage ist die notwendige Langfristigkeit und Offenheit des Standards sicherzustellen welche für die Verwendung in der Archivierung vorausgesetzt wird. Der PDF/A Standard ist aber nur ein Bauteil einer Lösung zur digitalen Langzeitarchivierung. Neben der wahl der Dateiformate gehören auch eine sorgfältige Planung und eine laufende Überprüfung der getroffenen Annahmen zu einer seriösen digitalen Langzeit Archivierung. Meiner Einschätzung nach ist das PDF/A Format eine solide Entscheidung für die digitale Langzeitarchivierung von Inhalten welche in Dokumentenform repräsentiert werden können. Mindestens genau so wichtig wie die Wahl des Dateiformats, sind aber auch das Etablieren von geigneten Prozessen innerhalb der Institution um die Langzeitarchvierung sicherzustellen.
Weiter ist es unerlässlich, die verschiedenen PDF Standards genau zu verstehen und nicht automatisch davon auszugehen, das alleine durch eine Konvertierung zu PDF/A-1b eine langfristige Lesbarkeit und Reproduzierbarkeit gewährleistet ist.

\section{Ausblick}
Die digitale Langzeitarchivierung muss sich stetig anpassen, da immer neue Inhalte mit neuen Formaten aufbewahrt werden müssen. Das bedeutet auch, dass der PDF Standard weiterentwicklet werden muss um zukünftigen Entwicklungen gerecht zu werden. In der nahen Zukunft werden immer meher und verschiedenartige Daten produziert werden welche als PDF aufbewahrt werden sollen. Hierbei wird der PDF Standard weitherhin einen wertvollen Beitrag leisten können. Durch die Veröffentlichung von weiteren Standards wie PDF/A-4, PDF/X-4 und PDF/VT wurden in der näheren Vergangenheit bereits neue PDF Standards veröffentlicht um sich ändernden Anforderungen gerecht zu werden. Durch Forschritte in der Forschung wird es wohl dazu kommen, dass der Standard um weitere Kompressionsalgorithmen und Bildformate ergänzt wird um mit dem steigenden Volumen Schritt zu halten.

%%%%%%%%%%%%%%%%%%%%%%%%%%%%%%%%%%%%%%%%%%%%%%%%%%%%%%%%%%%%%
%% LITERATUR UND ANDERE VERZEICHNISSE
%%%%%%%%%%%%%%%%%%%%%%%%%%%%%%%%%%%%%%%%%%%%%%%%%%%%%%%%%%%%%
%% Ein kleiner Abstand zu den Kapiteln im Inhaltsverzeichnis (toc)
\addtocontents{toc}{\protect\vspace*{\baselineskip}}

%% Abkürzungen
\cleardoublepage
\phantomsection
\addcontentsline{toc}{chapter}{Abkürzungen}
\chapter*{Abkürzungen}
\begin{acronym}[Abkürzungen]
\acro{PDF}{Portable Document Format}
\acro{PDF/A}{Portable Document Format for Archiving}
\acro{ISO}{International Organization for Standardization}
\end{acronym}

%% Literaturverzeichnis %%%%%%%%%%%%%%%%%%%%%%%%%%%%%%%%%%%%%%
%% ==> Eine Datei 'literatur.bib' wird hierfür benötigt.
\cleardoublepage
\phantomsection
\addcontentsline{toc}{chapter}{Literaturverzeichnis}
%\nocite{*} %Auch nicht-zitierte BibTeX-Einträge werden angezeigt.
\Urlmuskip=0mu plus 1mu\relax
\bibliographystyle{plainnat} %Art der Ausgabe: plain / apalike / amsalpha / ...
\bibliography{literatur} %Eine Datei 'literatur.bib' wird hierfür benötigt.
\end{document}